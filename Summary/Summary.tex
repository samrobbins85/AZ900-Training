\documentclass[addpoints]{exam}
\pagestyle{headandfoot}
\firstpageheadrule
\runningheadrule
\firstpageheader{AZ900}{}{Sam Robbins}
\runningheader{AZ900}{}{Sam Robbins}
\firstpagefooter{}{}{}
\runningfooter{}{}{}
\renewcommand{\solutiontitle}{\noindent\textbf{Solution:}\par\noindent}
\renewcommand{\familydefault}{\sfdefault}
\usepackage{listings}
\usepackage{amsmath}
\usepackage{amssymb}
\printanswers
\usepackage{graphicx}
\marksnotpoints
\bracketedpoints
\pointsdroppedatright
\pointsinrightmargin
\lstset{language=Python,
	basicstyle=\ttfamily,
	keywordstyle=\bfseries,
	showstringspaces=false,
	morekeywords={if, else, then, print, end, for, do, while,output},
	tabsize=4,
	mathescape=true,
	moredelim=**[is][\color{red}]{@}{@},
}
\begin{document}
\begin{center}
	\underline{\huge Questions likely to be on the exam}
\end{center}
\begin{questions}
\section{Describe Cloud concepts}
\subsection{Describe the benefits and considerations of using cloud services}
\question Describe terms such as High Availability, Scalability, Elasticity, Agility, Fault Tolerance, and Disaster Recovery
\begin{solution}[2in]
	High Availability - The ability of the application to continue running in a healthy state without significant downtime.\\
	Scalability - Adding more or removing resources when needed\\
	Elasticity - Automatically adding more resources to handle traffic\\
	Agility - Fast to set up\\
	Fault Tolerance - Able to continue running despite a fault, if one component fails, another takes over\\
	Disaster recovery - Able to recover data after an issue
\end{solution}
\question Describe the principles of economies of scale
\begin{solution}[2in]
	By purchasing and using in larger quantities, a discount is available, making costs less
\end{solution}
\question Describe the differences between Capital Expenditure (CapEx) and Operational Expenditure (OpEx)
\begin{solution}[2in]
	Capital expenditure is spending on equipment, such as servers. Operational expenditure is spending that continues throughout the usage of a product, such as power costs
\end{solution}
\question Describe the consumption-based model
\begin{solution}[2in]
	A user will only pay for what they use
\end{solution}
\subsection{Describe the differences between Infrastructure-as-a-Service (IaaS), Platform-as-a-Service (PaaS) and Software-as-a-Service (SaaS)}
\question Describe Infrastructure-as-a-Service (IaaS)
\begin{solution}[2in]
	Gives control over hardware, user is responsible for everything but the running of the hardware
\end{solution}

\question Describe Platform-as-a-Service (PaaS)
\begin{solution}[2in]
	Don't have to manage the infrastructure, provides an environment for building, testing and deploying software applications
\end{solution}

\question Describe Software-as-a-Service (SaaS)
\begin{solution}[2in]
	Only responsible for the data and access.
\end{solution}

\question Compare and contrast the three different service types
\begin{solution}[2in]
	All service types have no upfront costs.\\
	\\
	Software as a Service is cheapest, then platform, then infrastructure as the users have greater control.
\end{solution}
\subsection{Describe the differences between Public, Private and Hybrid cloud models}
\question Describe Public cloud
\begin{solution}[2in]
	No local hardware, all running on the cloud providers hardware
\end{solution}

\question Describe Private cloud
\begin{solution}[2in]
	Cloud set up in own datacenter
\end{solution}

\question Describe Hybrid cloud
\begin{solution}[2in]
	Combining public and private clouds
\end{solution}

\question Compare and contrast the three different cloud models
\begin{solution}[2in]
	Hybrid cloud is the most expensive, but also offers the greatest flexibility.\\
	\\
	Private and hybrid clouds are best for when compliance is important.\\
	\\
	Private and hybrid clouds also are good for when legacy systems need to keep going\\
	\\
	Public cloud is best for when scalability and agility is important.
\end{solution}

\section{Describe Core Azure Services}
\subsection{Describe the core Azure architectural components}
\question Describe Regions
\begin{solution}[2in]
	A geographical area on the planet containing at least one, but potentially multiple datacenters networked together
\end{solution}

\question Describe Availability Zones
\begin{solution}[2in]
	Physically separate datacenters within an Azure Region
\end{solution}

\question Describe Resource Groups
\begin{solution}[2in]
	A logical container for resources deployed on Azure
\end{solution}

\question Describe Azure Resource Manager
\begin{solution}[2in]
	A service for managing resources within Azure
\end{solution}

\question Describe the benefits and usage of core Azure architectural components
\begin{solution}[2in]
	Regions - Used to ensure the data is kept close to users to increase performance\\
	Availability Zones - Allow for backup to ensure data is kept online\\
	Resource groups - Allow for managing resource use\\
	Azure Resource Manager - Ensure the correct amount of resources are allocated to services
\end{solution}
\subsection{Describe some of the core products available in Azure}
\question Describe products available for compute
\begin{solution}[2in]
	Virtual Machines - Windows or Linux simulated machines\\
	Virtual Machine scale sets - Scaling for Azure Virtual Machines\\
	Azure Container Instances - Run containerised apps on Azure without provisioning servers or VMs\\
	Azure Kubernetes Service - Enables management of a cluster of VMs that run containerised services
\end{solution}

\question Describe products available for Networking
\begin{solution}[2in]
	Virtual Network - Connecting VMs to incoming VPN connections\\
	Load Balancer - Balances inbound and outbound connections to apps or service endpoints\\
	VPN gateway - Access Azure Virtual Networks through VPN gateways\\
	Application gateway - Optimizes app delivery while increasing security\\
	Content Delivery Network - Provides high bandwidth content to customers
\end{solution}

\question Describe products available for Storage
\begin{solution}[2in]
	Blob storage - Storage for very large objects\\
	Disk storage - Disks for Virtual Machines\\
	File storage - File shares you can access like a file storage\\
	Archive storage - Storage that is not accessed frequently
\end{solution}

\question Describe products available for Databases
\begin{solution}[2in]
	CosmosDB - NoSQL database\\
	Azure SQL Database - Fully managed relational database\\
	Azure Database for MySQL - Fully managed MySQL database\\
	Azure Database for PostgreSQL - Fully managed PostgreSQL database\\
	Azure Database Migration service - Migrate databases to the cloud
\end{solution}

\question Describe the Azure Marketplace and its usage scenarios
\begin{solution}[2in]
	Azure Marketplace allows you to purchase third party services which integrate with Azure
\end{solution}
\subsection{Describe some of the solutions available on Azure}
\question Describe IoT Products on Azure
\begin{solution}[2in]
	IoT Hub - Messaging hub for IoT devices\\
	IoT central - IoT Software as a Service
\end{solution}

\question Describe Big data and Analytics Products on Azure
\begin{solution}[2in]
	Azure Synapse Analytics - Fully managed data warehouse\\
	HDInsight - Process data with Hadoop in the cloud\\
	Azure Databricks - Collaborative Apache Spark based analytics
\end{solution}

\question Describe AI Products on Azure
\begin{solution}[2in]
	Azure Machine Learning Service - In depth machine learning\\
	Azure Machine Learning Studio - Drag and drop machine learning
\end{solution}

\question Describe Serverless Products on Azure
\begin{solution}[2in]
	Azure Functions - Event driven, serverless compute service\\
	Logic Apps - Execute workflows designed to automate business scenarios from blocks
	Event Grid - Want to do multiple things based on the information that is coming in
\end{solution}

\question Describe DevOps Products on Azure
\begin{solution}[2in]
	Azure DevOps - Git Repos and Kanban boards\\
	Azure DevTest Labs - Windows and Linux environments for testing
\end{solution}
\subsection{Describe Azure Management tools}
\question Describe Azure Tools
\begin{solution}[2in]
	Azure Portal - Access with any web browser\\
	Azure Powershell - Module for windows powershell or powershell core\\
	Azure CLI - Cross platform including with bash\\
	Cloud Shell - Browser accessible shell
\end{solution}
\section{Describe Security, Privacy, Compliance and Trust}
\subsection{Describe securing network connectivity in Azure}
\question Describe Network Security Groups
\begin{solution}[2in]
	This allows or denies inbound network traffic to your Azure Resources in a similar way to a firewall
\end{solution}
\question Describe Application Security Groups
\begin{solution}[2in]
	Allows you to group computers by name to manage them easier
\end{solution}
\question Describe User Defined Routes
\begin{solution}[2in]
	Static routes in Azure to override Azure's default system routes
\end{solution}
\question Describe Azure Firewall
\begin{solution}[2in]
	A firewall to protect Azure apps
\end{solution}
\question Describe Azure DDoS protection
\begin{solution}[2in]
	Protection for Azure Apps against DDoS attacks
\end{solution}
\subsection{Describe core Azure Identity Services}
\question Describe the difference between Authentication and Authorisation
\begin{solution}[2in]
	Authentication - Establish the identity of a person or service looking to access a resource\\
	Authorization - Establish what level of access an authenticated person or service has
\end{solution}
\question Describe Azure Active Directory
\begin{solution}[2in]
	Azure Active directory is a service to manage identity in Azure. Including Authentication.
\end{solution}
\question Describe Azure Multi Factor Authentication
\begin{solution}[2in]
	Authentication with something you know, something you are and something you have
\end{solution}
\subsection{Describe the security tools and features of Azure}
\question Describe Azure Security Center
\begin{solution}[2in]
	This provides security recommendations and monitors services for security vulnerabilities
\end{solution}
\question Describe Azure Key Vault
\begin{solution}[2in]
	This is a place to store secrets such as keys and passwords
\end{solution}
\question Describe Azure Information Protection
\begin{solution}[2in]
	This is a solution to classify and optionally protect documents by applying labels
\end{solution}
\question Describe Azure Advanced Threat Protection
\begin{solution}[2in]
	This is a solution to identify, protect and help to investigate threats
\end{solution}
\subsection{Describe Azure Governance Methodologies}
\question Describe Policies and Initiatives with Azure Policy
\begin{solution}[2in]
	Policy - A rule applied to a scope of resources\\
	Initiative - A group of policies to help track compliance
\end{solution}
\question Describe Role Based Access Control
\begin{solution}[2in]
	Giving users or services permissions based on roles provided to them
\end{solution}
\question Describe Locks
\begin{solution}[2in]
	This is a solution to prevent the accidental deletion of a resource
\end{solution}
\question Describe Azure Advisor security assistance
\begin{solution}[2in]
	Get proactive, actionable and personalised best practice recommendations
\end{solution}
\question Describe Azure Blueprints
\begin{solution}[2in]
	A declarative way to orchestrate the deployment of various resource templates and other artefacts
\end{solution}
\subsection{Describe monitoring and reporting options in Azure}
\question Describe Azure Monitor
\begin{solution}[2in]
	Maximising the availability and performance of applications using telemetry
\end{solution}
\question Describe Azure Service Health
\begin{solution}[2in]
	Provides guidance when issues with Azure affect you
\end{solution}

\question Describe the use cases and benefits of Azure Monitor and Azure Service Health
\begin{solution}[2in]
	Azure Monitor - Your configuration could be improved\\
	Azure Service Health - An Azure problem is going to affect your service
\end{solution}

\subsection{Describe privacy, compliance and data protection standards in Azure}

\question Describe industry compliance terms such as GDPR, ISO and NIST
\begin{solution}[2in]
	GDPR - Data stored has to have a reason and can be requested to be deleted\\
	ISO - Range of compliances, notably 27001 for information security management\\
	NIST - Government standards
\end{solution}

\question Describe the Microsoft Privacy Statement
\begin{solution}[2in]
	A statement saying what Microsoft is doing with you data in Azure
\end{solution}

\question Describe the Trust Center
\begin{solution}[2in]
	Details as to how Microsoft implements and supports security, privacy, compliance and transparency
\end{solution}

\question Describe the Service Trust Portal
\begin{solution}[2in]
	Hosts the compliance manager service and contains Microsoft audit reports
\end{solution}

\question Describe the Compliance Manager
\begin{solution}[2in]
	Allows you to ensure that you are compliant
\end{solution}

\question Describe Azure Government Cloud Services
\begin{solution}[2in]
	Azure services for the US Government with dedicated datacenters and different levels of compliance
\end{solution}

\question Describe Azure China cloud services
\begin{solution}[2in]
	Azure provided within China in partnership with 21Vianet
\end{solution}

\section{Describe Azure Pricing, Service Level agreements and Lifecycles}
\subsection{Describe Azure subscriptions}
\question Describe an Azure Subscription
\begin{solution}[2in]
	A logical container for resources in Azure
\end{solution}


\question Describe the uses and options with Azure subscriptions such access control and offer types
\begin{solution}[2in]
	Access Control - Subscriptions can be managed with RBAC so only certain users can access them \\
	Offer types - Monthly credits given in partnership with a Visual studio subscription
	
\end{solution}

\question Describe subscription management using Management groups
\begin{solution}[2in]
	Management groups allow you to order your Azure resources hierarchically into collections, which provide a further level of classification that is above the level of subscriptions
	
\end{solution}

\subsection{Describing planning and management of costs}

\question Describe options for Purchasing Azure products and services
\begin{solution}[2in]
	You can purchase directly on the web, enter an enterprise agreement or an agreement with a CSP (Cloud Solution Provider)
	
\end{solution}

\question Describe options around Azure Free account
\begin{solution}[2in]
	You can have a 12 month trial, including \$300 in credits, then some services are always free after that
 	
\end{solution}

\question Describe the factors affecting costs
\begin{solution}[2in]
	Resource types - Larger VMs will cost more etc\\
	Services - Your pricing depends on how you purchase Azure\\
	Locations - Some areas are more expensive than others\\
	Ingress and egress traffic - The more traffic you have, the greater the costs
	
\end{solution}

\question Describe Zones for billing purposes
\begin{solution}[2in]
	There are zones that it costs a certain amount to send data between
	
\end{solution}

\question Describe the pricing calculator
\begin{solution}[2in]
	This allows you to configure and estimate the costs for Azure products and features for your specific resources
	
\end{solution}

\question Describe the Total Cost of Ownership Calculator
\begin{solution}[2in]
	This allows you to estimate the savings you would get by transferring services from on premises to Azure
	
\end{solution}

\question Describe best practices for minimizing Azure costs
\begin{solution}[2in]
	Spending limits and quotas - Get notification when spending gets to a certain level\\
	Tags - Tag resources so you know how much projects cost\\
	Azure reservations - Reserve a resource for 1 or 3 years for a saving\\
	Azure Advisor Recommendations - Such as scaling down an underutilized VM
	
\end{solution}

\subsection{Describe Azure Service Level agreements}

\question Describe a Service Level Agreement
\begin{solution}[2in]
	A performance target to a product or service
	
\end{solution}

\question Describe Composite SLAs
\begin{solution}[2in]
	The SLA of a range of applications working together
	
\end{solution}

\subsection{Describe Service Lifecycle in Azure}
\question Describe Public and Private Preview features
\begin{solution}[2in]
	Public - Available to all users\\
	Private - Invitation only
	
\end{solution}

\question Describe General Availability
\begin{solution}[2in]
	When a feature is released as part of Azure's default product set after evaluation and testing
\end{solution}

\question Describe how to monitor feature updates and product changes
\begin{solution}[2in]
	Use the Azure updates page or the What's new link in the help menu
\end{solution}


\end{questions}




\end{document}
\documentclass{article}[18pt]
%\ProvidesPackage{format}
%Page setup
\usepackage[utf8]{inputenc}
\usepackage[margin=0.7in]{geometry}
\usepackage{parselines} 
\usepackage[english]{babel}
\usepackage{fancyhdr}
\usepackage{titlesec}
\hyphenpenalty=10000

\pagestyle{fancy}
\fancyhf{}
\rhead{Sam Robbins}
\rfoot{Page \thepage}

%Characters
\usepackage{amsmath}
\usepackage{amssymb}
\usepackage{gensymb}
\newcommand{\R}{\mathbb{R}}

%Diagrams
\usepackage{pgfplots}
\usepackage{graphicx}
\usepackage{tabularx}
\usepackage{relsize}
\pgfplotsset{width=10cm,compat=1.9}
\usepackage{float}

%Length Setting
\titlespacing\section{0pt}{14pt plus 4pt minus 2pt}{0pt plus 2pt minus 2pt}
\newlength\tindent
\setlength{\tindent}{\parindent}
\setlength{\parindent}{0pt}
\renewcommand{\indent}{\hspace*{\tindent}}

%Programming Font
\usepackage{courier}
\usepackage{listings}
\usepackage{pxfonts}

%Lists
\usepackage{enumerate}
\usepackage{enumitem}

% Networks Macro
\usepackage{tikz}


% Commands for files converted using pandoc
\providecommand{\tightlist}{%
	\setlength{\itemsep}{0pt}\setlength{\parskip}{0pt}}
\usepackage{hyperref}

% Get nice commands for floor and ceil
\usepackage{mathtools}
\DeclarePairedDelimiter{\ceil}{\lceil}{\rceil}
\DeclarePairedDelimiter{\floor}{\lfloor}{\rfloor}

% Allow itemize to go up to 20 levels deep (just change the number if you need more you madman)
\usepackage{enumitem}
\setlistdepth{20}
\renewlist{itemize}{itemize}{20}

% initially, use dots for all levels
\setlist[itemize]{label=$\cdot$}

% customize the first 3 levels
\setlist[itemize,1]{label=\textbullet}
\setlist[itemize,2]{label=--}
\setlist[itemize,3]{label=*}

% Definition and Important Stuff
% Important stuff
\usepackage[framemethod=TikZ]{mdframed}

\newcounter{theo}[section]\setcounter{theo}{0}
\renewcommand{\thetheo}{\arabic{section}.\arabic{theo}}
\newenvironment{important}[1][]{%
	\refstepcounter{theo}%
	\ifstrempty{#1}%
	{\mdfsetup{%
			frametitle={%
				\tikz[baseline=(current bounding box.east),outer sep=0pt]
				\node[anchor=east,rectangle,fill=red!50]
				{\strut Important};}}
	}%
	{\mdfsetup{%
			frametitle={%
				\tikz[baseline=(current bounding box.east),outer sep=0pt]
				\node[anchor=east,rectangle,fill=red!50]
				{\strut Important:~#1};}}%
	}%
	\mdfsetup{innertopmargin=10pt,linecolor=red!50,%
		linewidth=2pt,topline=true,%
		frametitleaboveskip=\dimexpr-\ht\strutbox\relax
	}
	\begin{mdframed}[]\relax%
		\centering
		}{\end{mdframed}}



\newcounter{lem}[section]\setcounter{lem}{0}
\renewcommand{\thelem}{\arabic{section}.\arabic{lem}}
\newenvironment{definition}[1][]{%
	\refstepcounter{lem}%
	\ifstrempty{#1}%
	{\mdfsetup{%
			frametitle={%
				\tikz[baseline=(current bounding box.east),outer sep=0pt]
				\node[anchor=east,rectangle,fill=blue!20]
				{\strut Definition};}}
	}%
	{\mdfsetup{%
			frametitle={%
				\tikz[baseline=(current bounding box.east),outer sep=0pt]
				\node[anchor=east,rectangle,fill=blue!20]
				{\strut Definition:~#1};}}%
	}%
	\mdfsetup{innertopmargin=10pt,linecolor=blue!20,%
		linewidth=2pt,topline=true,%
		frametitleaboveskip=\dimexpr-\ht\strutbox\relax
	}
	\begin{mdframed}[]\relax%
		\centering
		}{\end{mdframed}}
	
\newcounter{prob}[section]\setcounter{prob}{0}
\renewcommand{\theprob}{\arabic{section}.\arabic{lem}}
\newenvironment{problem}[1][]{%
	\refstepcounter{prob}%
	\ifstrempty{#1}%
	{\mdfsetup{%
			frametitle={%
				\tikz[baseline=(current bounding box.east),outer sep=0pt]
				\node[anchor=east,rectangle,fill=orange!20]
				{\strut Problem};}}
	}%
	{\mdfsetup{%
			frametitle={%
				\tikz[baseline=(current bounding box.east),outer sep=0pt]
				\node[anchor=east,rectangle,fill=orange!20]
				{\strut Problem:~#1};}}%
	}%
	\mdfsetup{innertopmargin=10pt,linecolor=orange!20,%
		linewidth=2pt,topline=true,%
		frametitleaboveskip=\dimexpr-\ht\strutbox\relax
	}
	\begin{mdframed}[]\relax%
	}{\end{mdframed}}
	
% Styling Pseudocode
\lstset{language=C,
	basicstyle=\ttfamily,
	keywordstyle=\bfseries,
	showstringspaces=false,
	morekeywords={if, else, then, print, end, for, do, while, Let},
	tabsize=4,
	mathescape=true,
	escapechar=£,
	numbers=left,
	stepnumber=1,
	frame=top,
	frame=bottom
}

\usepackage{caption}
\DeclareCaptionFormat{listing}{\rule{\dimexpr\textwidth+17pt\relax}{0.4pt}\par\vskip1pt#1#2#3}
\captionsetup[lstlisting]{format=listing,singlelinecheck=false, margin=0pt, font={sf},labelsep=space,labelfont=bf}


% Mathscr font
\usepackage{mathrsfs}

% Ensures there's a bit of the section before a new page
\preto{\subsection}{\Needspace{5\baselineskip}}
\preto{\section}{\Needspace{5\baselineskip}}
\lhead{AZ900}


\begin{document}
\begin{center}
\underline{\huge Azure Architecture and Service Guarantees}
\end{center}
\section{Datacenters and Regions}
\begin{definition}[Region]
A geographical area on the planet containing at least one, but potentially multiple datacenters networked together
\end{definition}
Note that some services or features are only available in certain regions
\subsection{Special Regions}
\begin{itemize}
	\item US DoD and Gov are separate datacenters specifically for government applications
	\item China datacenters are managed by 21Vianet
\end{itemize}
\section{Geographies}
Geographies are defined by geopolitical boundaries or county borders. This has the benefits:
\begin{itemize}
	\item Allow customers with specific compliance to keep their data close
	\item Ensure data residency, sovereignty, compliance and resiliency requirements are honoured within geographical boundaries.
	\item Ensure fault tolerance.
\end{itemize}
\begin{definition}[Data residency]
	The physical or geographic location of an organization's data or information.
\end{definition}
Azure has the following geographies:
\begin{itemize}
	\item Americas
	\item Europe
	\item Asia Pacific
	\item Middle Easy and Africa
\end{itemize}
\section{Availability Zones}
\begin{definition}[Availability Zone]
	Physically separate datacenters within an Azure region
\end{definition}
This means that if one zone goes down, the other continues working
\subsection{Using Availability Zones}
Azure services that support availability zones fall in two categories:
\begin{itemize}
	\item Zonal services - Pin the resource to a specific zone
	\item Zone-redundant services - Platform replicates automatically across zones
\end{itemize}
\section{Region Pairs}
\begin{definition}[Region Pair]
 	Another region within the same geography at least 300 miles away
\end{definition}
This reduces the likelihood of interruptions such as natural disasters.
\begin{important}
	An availability zone sits within a region pair. So an availability zone may be two datacenters in the UK, a region pair may be one in the UK and one in France
\end{important}
Advantages or region pairs:
\begin{itemize}
	\item If there's a large azure outage, one region out of every pair is prioritized to be restored as quickly as possible
	\item Planned Azure updates are rolled out to paired regions one region at a time to minimise downtime and risk of application outage
	\item Data continues to reside in the same geography as its pair for compliance purposes
\end{itemize}
\section{Service Level Agreements}
\begin{important}
	Azure does not provide SLAs for most services under the Free or Shared tiers
\end{important}
\subsection{Performance targets}
An SLA provides performance targets for a product or service
\subsection{Uptime and connectivity guarantees}
Typical SLA ranges from 99.9\% to 99.999\%
\subsection{Service credits}
Define how Microsoft will respond if a product fails to perform to its SLA spec.\\
\\
The lower the uptime percentage, the greater the service credit percentage given
\subsection{Composing SLAs across services}
Calculating the SLA of two services together is just multiplying them\\
\\
However if you install redundancy such as two databases, the SLA is calculated with
$$1-(\text{Failure chance of service 1} \times \text{Failure chance of service 2})$$
And this result is then multiplied by the SLA of the other component to get the SLA
\subsection{Improving App Reliability}
\begin{definition}[Resiliency]
	The ability of a system to recover from failures and continue to function
\end{definition}
\begin{definition}[Failure Mode Analysis]
	Identifying points of failure and defining how the application will respond to those failures
\end{definition}
\begin{definition}[Availability]
	The time that the system is functional and working
\end{definition}
Maximizing availability requires implementing measures to prevent service failures. However this can be difficult and expensive.\\
\\
You need to ensure that the minimum SLA in your system is one that you are happy with your service performing at.
\end{document}

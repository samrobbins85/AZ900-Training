\documentclass{article}[18pt]
%\ProvidesPackage{format}
%Page setup
\usepackage[utf8]{inputenc}
\usepackage[margin=0.7in]{geometry}
\usepackage{parselines} 
\usepackage[english]{babel}
\usepackage{fancyhdr}
\usepackage{titlesec}
\hyphenpenalty=10000

\pagestyle{fancy}
\fancyhf{}
\rhead{Sam Robbins}
\rfoot{Page \thepage}

%Characters
\usepackage{amsmath}
\usepackage{amssymb}
\usepackage{gensymb}
\newcommand{\R}{\mathbb{R}}

%Diagrams
\usepackage{pgfplots}
\usepackage{graphicx}
\usepackage{tabularx}
\usepackage{relsize}
\pgfplotsset{width=10cm,compat=1.9}
\usepackage{float}

%Length Setting
\titlespacing\section{0pt}{14pt plus 4pt minus 2pt}{0pt plus 2pt minus 2pt}
\newlength\tindent
\setlength{\tindent}{\parindent}
\setlength{\parindent}{0pt}
\renewcommand{\indent}{\hspace*{\tindent}}

%Programming Font
\usepackage{courier}
\usepackage{listings}
\usepackage{pxfonts}

%Lists
\usepackage{enumerate}
\usepackage{enumitem}

% Networks Macro
\usepackage{tikz}


% Commands for files converted using pandoc
\providecommand{\tightlist}{%
	\setlength{\itemsep}{0pt}\setlength{\parskip}{0pt}}
\usepackage{hyperref}

% Get nice commands for floor and ceil
\usepackage{mathtools}
\DeclarePairedDelimiter{\ceil}{\lceil}{\rceil}
\DeclarePairedDelimiter{\floor}{\lfloor}{\rfloor}

% Allow itemize to go up to 20 levels deep (just change the number if you need more you madman)
\usepackage{enumitem}
\setlistdepth{20}
\renewlist{itemize}{itemize}{20}

% initially, use dots for all levels
\setlist[itemize]{label=$\cdot$}

% customize the first 3 levels
\setlist[itemize,1]{label=\textbullet}
\setlist[itemize,2]{label=--}
\setlist[itemize,3]{label=*}

% Definition and Important Stuff
% Important stuff
\usepackage[framemethod=TikZ]{mdframed}

\newcounter{theo}[section]\setcounter{theo}{0}
\renewcommand{\thetheo}{\arabic{section}.\arabic{theo}}
\newenvironment{important}[1][]{%
	\refstepcounter{theo}%
	\ifstrempty{#1}%
	{\mdfsetup{%
			frametitle={%
				\tikz[baseline=(current bounding box.east),outer sep=0pt]
				\node[anchor=east,rectangle,fill=red!50]
				{\strut Important};}}
	}%
	{\mdfsetup{%
			frametitle={%
				\tikz[baseline=(current bounding box.east),outer sep=0pt]
				\node[anchor=east,rectangle,fill=red!50]
				{\strut Important:~#1};}}%
	}%
	\mdfsetup{innertopmargin=10pt,linecolor=red!50,%
		linewidth=2pt,topline=true,%
		frametitleaboveskip=\dimexpr-\ht\strutbox\relax
	}
	\begin{mdframed}[]\relax%
		\centering
		}{\end{mdframed}}



\newcounter{lem}[section]\setcounter{lem}{0}
\renewcommand{\thelem}{\arabic{section}.\arabic{lem}}
\newenvironment{definition}[1][]{%
	\refstepcounter{lem}%
	\ifstrempty{#1}%
	{\mdfsetup{%
			frametitle={%
				\tikz[baseline=(current bounding box.east),outer sep=0pt]
				\node[anchor=east,rectangle,fill=blue!20]
				{\strut Definition};}}
	}%
	{\mdfsetup{%
			frametitle={%
				\tikz[baseline=(current bounding box.east),outer sep=0pt]
				\node[anchor=east,rectangle,fill=blue!20]
				{\strut Definition:~#1};}}%
	}%
	\mdfsetup{innertopmargin=10pt,linecolor=blue!20,%
		linewidth=2pt,topline=true,%
		frametitleaboveskip=\dimexpr-\ht\strutbox\relax
	}
	\begin{mdframed}[]\relax%
		\centering
		}{\end{mdframed}}
	
\newcounter{prob}[section]\setcounter{prob}{0}
\renewcommand{\theprob}{\arabic{section}.\arabic{lem}}
\newenvironment{problem}[1][]{%
	\refstepcounter{prob}%
	\ifstrempty{#1}%
	{\mdfsetup{%
			frametitle={%
				\tikz[baseline=(current bounding box.east),outer sep=0pt]
				\node[anchor=east,rectangle,fill=orange!20]
				{\strut Problem};}}
	}%
	{\mdfsetup{%
			frametitle={%
				\tikz[baseline=(current bounding box.east),outer sep=0pt]
				\node[anchor=east,rectangle,fill=orange!20]
				{\strut Problem:~#1};}}%
	}%
	\mdfsetup{innertopmargin=10pt,linecolor=orange!20,%
		linewidth=2pt,topline=true,%
		frametitleaboveskip=\dimexpr-\ht\strutbox\relax
	}
	\begin{mdframed}[]\relax%
	}{\end{mdframed}}
	
% Styling Pseudocode
\lstset{language=C,
	basicstyle=\ttfamily,
	keywordstyle=\bfseries,
	showstringspaces=false,
	morekeywords={if, else, then, print, end, for, do, while, Let},
	tabsize=4,
	mathescape=true,
	escapechar=£,
	numbers=left,
	stepnumber=1,
	frame=top,
	frame=bottom
}

\usepackage{caption}
\DeclareCaptionFormat{listing}{\rule{\dimexpr\textwidth+17pt\relax}{0.4pt}\par\vskip1pt#1#2#3}
\captionsetup[lstlisting]{format=listing,singlelinecheck=false, margin=0pt, font={sf},labelsep=space,labelfont=bf}


% Mathscr font
\usepackage{mathrsfs}

% Ensures there's a bit of the section before a new page
\preto{\subsection}{\Needspace{5\baselineskip}}
\preto{\section}{\Needspace{5\baselineskip}}
\lhead{AZ900}


\begin{document}
\begin{center}
\underline{\huge Azure Networking Options}
\end{center}
\section{Deploying your site to Azure}
\subsection{Using an N tier architecture}
An architectural pattern that can be used to build loosely couples systems is N-tier\\
\\
This divides an architecture into two or more logical tiers. Architecturally, a higher tier can access services from a lower tier, but a lower tier should never access a higher tier.\\
\\
Tiers help separate concerns and ideally are designed to be reusable. Using a tiered architecture also simplifies maintenance. Tiers can be updated or replaced independently, and new tiers can be inserted if needed.
\subsection{Virtual Network}
\begin{definition}[Virtual Network]
	A logically isolated network on Azure
\end{definition}
A virtual network allows Azure resources to securely communicate with each other, the internet and on premises networks.\\
\\
A virtual network is scoped to a single region, however, multiple virtual networks form different regions can be connected together using virtual network peering. \\
\\
Virtual networks can be segmented into one or more subnets to allow you to organize and secure your resources into discrete sections.\\
\\
For VMs that the users interact with directly, such as the web tier, the VM will have both a public and private IP whereas other tiers will just have private IPs.
\begin{definition}[VPN Gateway]
	Allows a secure connection between an Azure Virtual Network and an on premises location over the internet
\end{definition}
\subsection{Network security group}
This allows or denies inbound network traffic to your Azure resources in a similar way to a firewall.
\section{Azure Load Balancer}
\begin{definition}[Availability]
	How long your service is up and running without interruption
\end{definition}
\begin{definition}[Resiliency]
	A system's ability to stay operational during abnormal conditions
\end{definition}
\begin{definition}[Load Balancer]
	A way of distributing traffic evenly among each system in a pool
\end{definition}
The user connects to the load balancer, which then decides which Virtual Machine will process the request. This also allows you to run maintenance tasks without interrupting service as the load balancer detects that the VM is unresponsive and directs traffic to other VMs in the pool. 
\subsection{What is Azure load balancer?}
You can use it with:
\begin{itemize}
	\item Incoming internet traffic
	\item Internal traffic across Azure services
	\item Port forwarding for specific traffic
	\item Outbound connectivity for VMs in your virtual network
\end{itemize}
With Azure load balancer there's no infrastructure or software to maintain, just a set of rules.
\subsection{Azure Application Gateway}
Azure Application gateway is a load balancer for web applications. It uses Azure Load balancer at the transport level but uses routing rules for more advanced scenarios.\\
\\
Benefits over a load balancer:
\begin{itemize}
	\item Cookie affinity
	\item SSL termination
	\item Web application firewall
	\item URL rule based routes
	\item Rewrite HTTP headers
\end{itemize}
\section{Azure Traffic Manager}
\begin{definition}[Network latency]
	The time it takes for data to travel over the network
\end{definition}
One way to improve network latency is to scale out exact copies of your service to more than one region\\
\\
Azure Traffic Manager uses the DNS server that's closest to the user to direct user traffic to a globally distributed endpoint.
\end{document}
